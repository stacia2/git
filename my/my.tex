% !Mode:: "TeX:UTF-8"
\documentclass{xcumcmart}
%\usepackage{subfigure}
% \title{text}这里是显示在第三页的文章标题
\title{基于热传导模型的防热服设计}
% \displaytitle{text}这里是显示在承诺书上的文章标题,注意,不能换行,如果题目特别长,要进行适当的缩写
\displaytitle{A}
% \school{text}命令用于在承诺书上显示学校名称。按要求,此处应填写全称
\school{}
% 以下命令分别显示队员及指导教师姓名
\authorone{}
\authortwo{}
\authorthree{}
\advisor{}

\usepackage{metalogo,hyperref} % 这里加载的宏包仅仅是为了本示例文档,实际使用时可以根据需要删除。
\usepackage[superscript,compress,sort]{cite}
\makeatletter
\def\@citess#1{\textsuperscript{[#1]}}
\usepackage[perpage]{footmisc}  %每页脚注重新编号
\usepackage{subfigure}
\usepackage{listings}
\lstset{language=Matlab}
\lstset{breaklines}
\lstset{extendedchars=false}
%\lstset{numbers=left, numberstyle=\small}
\lstset{keywordstyle=\bfseries,commentstyle=\color{gray}}
\lstset{frame=lines}
\usepackage{array}
\usepackage{multirow}
\usepackage{booktabs}
\usepackage{threeparttable}
\usepackage{caption}
\usepackage{paralist}
\usepackage{mathrsfs}
\usepackage{amsmath}
\usepackage{amsfonts}
\usepackage{algorithm}  
\usepackage{algorithmicx}
\usepackage{algpseudocode}  
\usepackage{graphicx}
\usepackage{float}
\usepackage{diagbox}
\begin{document}
\begin{minipage}{0.9\textwidth}
\centering\LARGE\textbf{基于热传导模型的防热服设计}
\end{minipage}

\par
\par\par
\renewcommand{\abstractname}{摘\quad 要} 
\begin{abstract}
在高温环境下工作时,需要穿着专用防护服,以避免高温对人体的伤害。本题意在考察材料厚度不同的情况下,防热服对热量的阻隔能力,进而对材料的厚度进行调整,以使防热服能达到预期的效果,高温作业可顺利开展。\par
\textbf{针对问题一}\quad 建立\textbf{热传导}模型,认为热量从服装表面传至人体的过程可简化为一维情况。将皮肤认为是\textbf{第五层}材料,其厚度和热扩散率待定。确定模型的初始条件和边界条件,求解微分方程。考察所求数值解与给定实验数据的误差,用内点法优化皮肤参数,得到皮肤厚度$21.92$mm,热扩散率$1.243\mathrm{m^2/s}$,并求解出温度的时空分布图。\par
\textbf{针对问题二}\quad 沿用问题一的模型,确定最小化II层厚度为目标函数,在问题二所给的体表温度上限等相关条件约束下进行优化。确定温度随时间变化的单调性,化简约束条件,并利用\textbf{二分查找}快速寻找最优解,得到II层最优厚度为$8.040$mm。\par
\textbf{针对问题三}\quad 该问题有两个待优化变量:II层和IV层厚度。确定防热服\textbf{总质量最小}为优化目标,建立人体模型以量化总质量的评估,即得到总质量与各层厚度之间的函数关系式。由于在IV层厚度确定的情况下,优化方法与问题二相同,因此可化归为对IV层空气层厚度的优化。得到结果II层和IV层的最优厚度分别为$15.42$mm和$0.6$mm,防热服总质量$28.13$kg。\par

\textbf{关键词:\quad 热传导\quad 约束优化\quad 二分查找}
\end{abstract}
\end{document}